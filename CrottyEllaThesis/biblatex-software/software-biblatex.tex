\documentclass{ltxdockit}
\usepackage{a4wide}
\usepackage{hyperref}
\usepackage[T1]{fontenc}
\usepackage[utf8]{inputenc}
\usepackage{alltt}
\usepackage{listings}
\usepackage{shortvrb}
\MakeShortVerb{\|}

\titlepage{%
  title={Bib\LaTeX\ style extension for Software},
  subtitle={Citation and References macros for Bib\LaTeX},
  url={https://www.ctan.org/tex-archive/macros/latex/contrib/biblatex-contrib/biblatex-software},
  author={Roberto Di Cosmo},
  email={roberto@dicosmo.org},
  revision={1.2-6},
  date={\today}}

\hypersetup{%
  pdftitle={Bib\LaTeX\ style extension for Software},
  pdfsubject={Style for Bib\LaTeX},
  pdfauthor={Roberto Di Cosmo},
  pdfkeywords={latex, biblatex, software, style}}

\begin{document}

\printtitlepage
\tableofcontents

\section{Introduction}\label{sec:intro}

\subsection{About}

Software plays a significant role in modern research, and it must be properly
acknowledged and referenced in scholarly works. To this end, specific
bibliographic entries for describing \emph{software}, \emph{software versions},
\emph{software modules} and \emph{code fragments} have been designed by a
dedicated task force at Inria in 2020 that brought together researchers from
Computer Science and Applied Mathematics to discuss best practices for archiving
and referencing software source code~\cite{2020GtCitation}.\\

This package is a Bib\LaTeX\ \emph{style extension} that adds support for these
four \emph{software entry types} to any other Bib\LaTeX\ style used in documents
typeset in \latex. It is made up of the following key components: a references
section style (\path{software.bbx}), a data model extension
(\path{software.dbx}) and string localisation files
(\path{<language>-software.lbx})\footnote{String localisations are provided for
  some languages: localisations for other languages are welcome, feel free to
  contribute them on the official repository, see Section \emph{Contributing} below.}.\\

The distribution contains other material, for demonstration purposes, and for
more advanced use.

\subsection{License}

Permission is granted to copy, distribute and/or modify this software under
the terms of the \latex Project Public License, version
1.3c\footnote{\url{http://www.latex-project.org/lppl.txt}}. The current
maintainer is Roberto Di Cosmo (\textcopyright\ 2020-2024).

\subsection{History}

When I decided to start the Software Heritage initiative in 2015, software in
general and research software in particular was not yet a first class citizen in
the scholarly world. The absence of support for properly citing software in a
bibliography was just one of the many signs of this lack of recognition.

In order to properly reference a software project, and ensure that these
references are stable enough to pass the test of time, it was necessary to build
a \emph{universal archive} for software source code, and to equip every software
artifact with \emph{intrinsic} identifiers.

Now that \href{https://www.softwareheritage.org}{Software Heritage} is providing
the universal archive~\cite{swhcacm2018}, with Software Heritage \emph{intrinsic
  identifiers} (SWH-IDs) widely available~\cite{cise-2020-doi}, it is finally
possible to propose proper bibliographic entries for software, at various levels
of granularity, down to the line of code.

\subsection{Acknowledgments}

Thanks to the Inria working group members for their precious feedback and
contribution to the desing of the software bibliography entries: Pierre Alliez,
Benjamin Guedj, Alain Girault, Morane Gruenpeter, Mohand-Said Hacid, Arnaud
Legrand, Xavier Leroy, Nicolas Rougier and Manuel Serrano.

\section{Software entries}

There are four entry types, corresponding to different granularities
in the identification of the (part of) software artifacts that one
whishes to cite. They are listed below in order of granularity.

\subsection{software}
Computer software. 

\emph{Required fields:} \texttt{author} / \texttt{editor}, \texttt{title}, \texttt{url}, \texttt{year}

\emph{Optional fields:} \texttt{abstract}, \texttt{date}, \texttt{doi}, \texttt{eprint}, \texttt{eprintclass}, \texttt{eprinttype}, \texttt{file}, \texttt{hal\_id}, \texttt{hal\_version},
\texttt{institution}, \texttt{license}, \texttt{month}, \texttt{note}, \texttt{organization}, \texttt{publisher},
\texttt{related}, \texttt{relatedtype}, \texttt{relatedstring}, \texttt{repository}, \texttt{swhid}, \texttt{urldate}, \texttt{version}

\subsection{softwareversion}
A specific version of a software.  Inherits values of missing
fields from the entry mentioned in the \texttt{crossref} field.

\emph{Required fields:} \texttt{author} / \texttt{editor}, \texttt{title}, \texttt{url}, \texttt{version}, \texttt{year}

\emph{Optional fields:} \texttt{abstract}, \texttt{crossref}, \texttt{date}, \texttt{doi}, \texttt{eprint}, \texttt{eprintclass}, \texttt{eprinttype}, \texttt{file}, \texttt{hal\_id}, \texttt{hal\_version},
\texttt{institution}, \texttt{introducedin}, \texttt{license}, \texttt{month}, \texttt{note}, \texttt{organization}, \texttt{publisher},
\texttt{related}, \texttt{relatedtype}, \texttt{relatedstring}, \texttt{repository}, \texttt{swhid}, \texttt{subtitle}, \texttt{urldate}

\subsection{softwaremodule}

A specific module of a larger software project. Inherits values of missing
fields from the entry mentioned in the \texttt{crossref} field.

\emph{Required fields:} \texttt{author}, \texttt{subtitle}, \texttt{url}, \texttt{year}

\emph{Optional fields:} \texttt{abstract}, \texttt{crossref}, \texttt{date}, \texttt{doi}, \texttt{eprint}, \texttt{eprintclass}, \texttt{eprinttype}, \texttt{editor}, \texttt{file}, \texttt{hal\_id}, \texttt{hal\_version},
\texttt{institution}, \texttt{introducedin}, \texttt{license}, \texttt{month}, \texttt{note}, \texttt{organization}, \texttt{publisher},
\texttt{related}, \texttt{relatedtype}, \texttt{relatedstring}, \texttt{repository}, \texttt{swhid}, \texttt{title}, \texttt{urldate}, \texttt{version}

\subsection{codefragment}

A code fragment (e.g. a specific algorithm in a program or library).
Inherits values of missing fields from the entry mentioned in the \texttt{crossref} field.

\emph{Required fields:}  \texttt{url}

\emph{Optional fields:} \texttt{author}, \texttt{abstract}, \texttt{crossref}, \texttt{date}, \texttt{doi}, \texttt{eprint}, \texttt{eprintclass}, \texttt{eprinttype}, \texttt{file}, \texttt{hal\_id}, \texttt{hal\_version},
\texttt{institution}, \texttt{introducedin}, \texttt{license}, \texttt{month}, \texttt{note}, \texttt{organization}, \texttt{publisher},
\texttt{related}, \texttt{relatedtype}, \texttt{relatedstring}, \texttt{repository}, \texttt{swhid}, \texttt{subtitle}, \texttt{title}, \texttt{urldate}, \texttt{version}, \texttt{year}

The \textbf{softwareversion}, \textbf{softwaremodule} and \textbf{codefragment} entries can inherit
the missing fields from another entry designated by the \emph{crossref} field, which
is expected to be higher in the granularity hierarchy: \textbf{softwareversion} may
inherit from a \textbf{software} entry, \textbf{softwaremodule} may inherit from a
\textbf{softwareversion} or a \textbf{software} entry, and \textbf{codefragment} may inherit from all
other entries.

\section{Field description}
The field description is based on the \href{http://mirrors.ibiblio.org/CTAN/macros/latex/exptl/biblatex/doc/biblatex.pdf}{biblatex documentation}
\subsection{Data fields}
\begin{description}
\item[{abstract}] field (literal). This field is intended for recording 
abstracts in a bib file, to be printed by a special bibliography 
style. It is not used by all standard bibliography styles.
\item[{author}] list (name). The authors of the title.
\item[{date [biblatex only]}] field (date). The date of creation or release in ISO format.
\item[{editor}] list (name). The coordinator(s) of large modular software projects.
\item[{file}] field (verbatim). A link to download a copy of the work.
\item[{doi}] field (verbatim). The Digital Object Identifier of the work.
\item[{eprint [biblatex only]}] field (verbatim). An electronic identifier of the work. This field can be used to accommodate electronic identifiers different from the ones that have a dedicate field in this style.
\item[{eprinttype [biblatex only]}] field (verbatim). The type of eprint identifier, e. g., the name of the archive, repository, service, or system the eprint field refers to. Will be typeset by default as a prefix of the content of the eprint field.
\item[{eprintclass [biblatex only]}] field (verbatim). Additional information related to the resource indicated by the eprinttype field. This could be a section of an archive, a path indicating a service, a classification of some sort.
\item[{\texttt{hal\_id} [not in biblatex standard styles]}] field (verbatim). A digital identifier for the 
software record including its description and metadata on HAL.
\item[{\texttt{hal\_version} [not in biblatex standard styles]}] field (verbatim). The version of the HAL software record designated by \texttt{hal\_id}.
\item[{license [not in biblatex standard styles]}] list (literal). The license/s of the title 
in SPDX format.
\item[{month}] field (literal). The month of creation or release. 
In BibLaTeX, this must be an integer, not an ordinal or a string.
For compatibility with BibTeX, one can also use the three letter
abbreviations \emph{jan, feb, mar, apr, may, jun, jul, aug, sep, oct, nov, dec},
which must be given without any braces or quotes.
\item[{note}] field (literal). Release note of the cited version.
\item[{institution}] list (literal). The institution(s) that took part in the software project.
\item[{introducedin [not in biblatex standard styles]}] field (literal). If this is a software module or fragment,
the version of the containing project where it has been first introduced.
\item[{organization}] list (literal). The organization(s) that took part in the software project.
\item[{publisher}] list (literal). The name(s) of the publisher(s) of the \emph{qualified} software record.
\item[{related [biblatex only]}] field (separated values). Citation keys of other entries which have a relationship to this entry.
\item[{relatedtype [biblatex only]}] field (identifier).
\item[{relatedstring [biblatex only]}] field (literal).
\item[{repository [not in biblatex standard styles]}] field (uri). The url of the code repository (e.g on GitHub, GitLab).
\item[{swhid [not in biblatex standard styles]}] field (verbatim). The identifier of the digital
object (a.k.a the software artifact itself). The intrinsic identifier
of the item is an swh-id (swh:cnt for a content, swh:dir for a directory, swh:rev for
a revision, swh:rel for a release, etc.). See \href{https://docs.softwareheritage.org/devel/swh-model/persistent-identifiers.html}{the SWH-ID specification}.
\item[{subtitle}] field (literal). The title of a component of the software artifact.
\item[{title}] field (literal). The title of the software artifact.
\item[{url}] field (uri). The url of a reference resource (e.g the project's official webpage).
\item[{urldate}] field (date). The access date of the address specified in the url field.
\item[{version}] field (literal). The revision number of a piece of software, a manual, etc.
\item[{year}] field (literal). The year of creation or release.
\end{description}
\subsection{Special fields}
\begin{description}
\item[{crossref}] field (entry key). This field holds an entry key for the
cross-referencing feature. Child entries with a crossref field inherit data
from the parent entry specified in the crossref field.
\end{description}

\section{Examples}

Here are a few example of use of the proposed entries.

\subsection{software and softwareversion}

This is an example description of a software release using a single \texttt{@softwareversion} entry.

\begin{verbatim}
@softwareversion {delebecque:hal-02090402-condensed,
  title = {Scilab},
  author = {Delebecque, Fran{\c c}ois and Gomez, Claude and Goursat, Maurice 
    and Nikoukhah, Ramine and Steer, Serge and Chancelier, Jean-Philippe},
  url = {https://www.scilab.org/},
  date = {1994-01},
  file = {https://hal.inria.fr/hal-02090402/file/scilab-1.1.tar.gz},
  institution = {Inria},
  license = {Scilab license},
  hal_id = {hal-02090402},
  hal_version = {v1},
  swhid = {swh:1:dir:1ba0b67b5d0c8f10961d878d91ae9d6e499d746a;
	   origin=https://hal.archives-ouvertes.fr/hal-02090402},
  version = {1.1},
  note = {First Scilab version. It was distributed by anonymous ftp.},
  repository= {https://github.com/scilab/scilab},
  abstract = {Software for Numerical Computation freely distributed.}
}
\end{verbatim}

The same information can also be represented using a \texttt{@software} / \texttt{@softwareversion}
pair that factors out the general information in the \texttt{@software} entry, so for
other versions only the changes need to be added in a new \texttt{@softwareversion} entry:

\begin{verbatim}
@software {delebecque:hal-02090402,
  title = {Scilab},
  author = {Delebecque, Fran{\c c}ois and Gomez, Claude and Goursat, Maurice 
    and Nikoukhah, Ramine and Steer, Serge and Chancelier, Jean-Philippe},
  date = {1994},
  institution = {Inria},
  license = {Scilab license},
  hal_id = {hal-02090402},
  hal_version = {v1},
  url = {https://www.scilab.org/},
  abstract = {Software for Numerical Computation freely distributed.},
  repository= {https://github.com/scilab/scilab},
}

@softwareversion {delebecque:hal-02090402v1,
  version = {1.1},
  date = {1994-01},
  file = {https://hal.inria.fr/hal-02090402/file/scilab-1.1.tar.gz},
  swhid = {swh:1:dir:1ba0b67b5d0c8f10961d878d91ae9d6e499d746a;
	   origin=https://hal.archives-ouvertes.fr/hal-02090402},
  note = {First Scilab version. It was distributed by anonymous ftp.},
  crossref = {delebecque:hal-02090402}
}
\end{verbatim}

\subsection{softwaremodule}

For highly modular software projects, like CGAL, one may need to 
reference specifically a particular module, that has distinguished
authors, and may heve been introduced in the project at a later time.


The following example uses \href{https://doc.cgal.org/latest/Manual/how\_to\_cite\_cgal.bib}{the informations in the existing BibTeX entries for CGAL}
that currently refer to the user manual, to create the corresponding software entries.

\begin{verbatim}
@software {cgal,
 title = {The Computational Geometry Algorithms Library},
 author = {{The CGAL Project}},
 editor = {{CGAL Editorial Board}},
 date = {1996},
 url = {https://cgal.org/}
}

@softwareversion{cgal:5-0-2,
 crossref = {cgal},
 version = {{5.0.2}},
 url = {https://docs.cgal.org/5.02},
 date = {2020},
 swhid = {swh:1:rel:636541bbf6c77863908eae744610a3d91fa58855;
	  origin=https://github.com/CGAL/cgal/}
}

@softwaremodule{cgal:lp-gi-20a,
 crossref = {cgal:5-0-2},
 author = {Menelaos Karavelas},
 subtitle = {{2D} Voronoi Diagram Adaptor},
 license = {GPL},
 introducedin = {cgal:3-1},
 url = {https://doc.cgal.org/5.0.2/Manual/packages.html#PkgVoronoiDiagram2},
}
\end{verbatim}

Of course, it is always be possible to use only one entry to get an equivalent
result; here one would use just \texttt{@softwaremodule} with all the needed data
fields as follows:

\begin{verbatim}
@softwaremodule{cgal:lp-gi-20a-condensed,
 title = {The Computational Geometry Algorithms Library},
 subtitle = {{2D} Voronoi Diagram Adaptor},
 author = {Menelaos Karavelas},
 editor = {{CGAL Editorial Board}},
 license = {GPL},
 version = {{5.0.2}},
 introducedin = {cgal:3-1},
 date = {2020},
 swhid = {swh:1:rel:636541bbf6c77863908eae744610a3d91fa58855;
	  origin=https://github.com/CGAL/cgal/},
 url = {https://doc.cgal.org/5.0.2/Manual/packages.html#PkgVoronoiDiagram2},
}
\end{verbatim}

\subsection{codefragment}

Finally, if one wants to have a particular code fragment appear in the bibliography,
we can do this as follows:
\begin{verbatim}
@software {parmap,
  title = {The Parmap library},
  author = {Di Cosmo, Roberto and Marco Danelutto},
  date = {2012},
  institution = {{Inria} and {University of Paris} and {University of Pisa}},
  license = {LGPL-2.0},
  url = {https://rdicosmo.github.io/parmap/},
  repository= {https://github.com/rdicosmo/parmap},
}

@softwareversion {parmap-1.1.1,
  crossref = {parmap},
  date = {2020},
  version = {1.1.1},
  swhid = {swh:1:rel:373e2604d96de4ab1d505190b654c5c4045db773;
     origin=https://github.com/rdicosmo/parmap;
     visit=swh:1:snp:2a6c348c53eb77d458f24c9cbcecaf92e3c45615},
}

@codefragment {simplemapper,
  subtitle = {Core mapping routine},
  swhid = {swh:1:cnt:43a6b232768017b03da934ba22d9cc3f2726a6c5;
     origin=https://github.com/rdicosmo/parmap;
     visit=swh:1:snp:2a6c348c53eb77d458f24c9cbcecaf92e3c45615;
     anchor=swh:1:rel:373e2604d96de4ab1d505190b654c5c4045db773;
     path=/src/parmap.ml;
     lines=192-228},
  crossref = {parmap-1.1.1}
}
\end{verbatim}

Of course, it is always be possible to use only one entry to get an equivalent
result; here one would use just \texttt{@codefragment} with all the needed data
fields as follows:

\begin{verbatim}
@codefragment {simplemapper-condensed,
  title = {The Parmap library},
  author = {Di Cosmo, Roberto and Marco Danelutto},
  date = {2020},
  institution = {{Inria} and {University of Paris} and {University of Pisa}},
  license = {LGPL-2.0},
  url = {https://rdicosmo.github.io/parmap/},
  repository= {https://github.com/rdicosmo/parmap},
  version = {1.1.1},
  subtitle = {Core mapping routine},
  swhid = {swh:1:cnt:43a6b232768017b03da934ba22d9cc3f2726a6c5;
     origin=https://github.com/rdicosmo/parmap;
     visit=swh:1:snp:2a6c348c53eb77d458f24c9cbcecaf92e3c45615;
     anchor=swh:1:rel:373e2604d96de4ab1d505190b654c5c4045db773;
     path=/src/parmap.ml;
     lines=192-228}
}
\end{verbatim}


\section{Use}\label{ref:use}
\label{use}

This package can be used as a standalone on the fly extension, or to produce
full bibliographic styles that extend pre-existing styles.

\subsection{Use as an \emph{on the fly} extension}

The simplest way to use this package is to follow the example given in the
\path{sample-use-sty.tex} that shows how one can \emph{extend on the fly} any
existing Bib\LaTeX\ style by just doing the following:

\begin{itemize}
\item pass the \texttt{datamodel=software} option to the \texttt{biblatex} package as in
 \begin{ltxcode}
   \usepackage[datamodel=software]{biblatex}
 \end{ltxcode}
 \item load the software biblatex style using
 \begin{ltxcode}
   \usepackage{software-biblatex}
 \end{ltxcode}
 \item set software specific bibliography options using the macro |\ExecuteBibliographyOptions|;
   the options with their default values are as in
 \begin{ltxcode}
   \ExecuteBibliographyOptions{
     halid=true,
     swhid=true,
     shortswhid=false,
     swlabels=true,
     vcs=true,
     license=true}
 \end{ltxcode}
\end{itemize}

This is quite handy to add support for software entries in a single article, as
it is enough to add \path{software.dbx}, \path{software.bbx},
\path{<language>-software.lbx} and \path{software-biblatex.sty} to make it work.

\subsection{Generate extended styles}

When a more systematic use is foreseen, as for institution-wide reports, or
conference and journal proceedings, it is more appropriate to generate a new
biblatex style that includes support for the software entries right away.

The following simple mechanism is provided for this use case:

\begin{itemize}
\item add to the \path{stublist} file the names of all the existing styles that must be extended
\item run \texttt{make biblatex-styles} to produce new style files, with an added \texttt{+sw} suffix,
   for each of the existing style
\item install the newly generated files in the standard path where Bib\LaTeX\ files are found
\end{itemize}

The stock \path{stublist} file contains the names of all the standard Bib\LaTeX\ 
styles.  If this approach is followed, then one can load directly the extended
file, and the software specific bibliography options become available when
loading the Bib\LaTeX\ package directly.  See the \path{sample.tex} file for a
working example.

\subsection{Installation}\label{sec:install}

This package may become available in standard distributions like \TeX Live as
|biblatex-software|. To install manually, you can download it from CTAN and
then, put the relevant files in your texmf tree, usually:\\

\noindent\path{<texmf>/tex/latex/biblatex-software/software-biblatex.sty}\\
\path{<texmf>/tex/latex/biblatex-software/software.bbx}\\
\path{<texmf>/tex/latex/biblatex-software/software.dbx}\\
\path{<texmf>/tex/latex/biblatex-software/<language>-software.lbx}\\


\subsection{Package options}\label{sec:options}

The following options are available to control the typesetting of
software related entries.

\begin{ltxcode}
  swlabels=true|false
\end{ltxcode}

\noindent Software is a special research output, distinct from
publications, hence software entries in a bibliography are
distinguished by a special label by default.
This behaviour can be disabled by setting this option to |false|.

\begin{ltxcode}
  license=true|false
\end{ltxcode}

\noindent This option controls whether license information for
the software entry is typeset. The default is |true|.

\begin{ltxcode}
  halid=true|false
\end{ltxcode}

\noindent This option controls the inclusion of the identifier on the HAL repository of the
metadata record for the software described in the entry. The default is |true|.

\begin{ltxcode}
  swhid=true|false
\end{ltxcode}

\noindent This option controls the inclusion of the identifier on the Software Heritage archive
(SWHID) of the source code of the software described in the entry. The default is |true|.

\begin{ltxcode}
  shortswhid=true|false
\end{ltxcode}

\noindent This option controls the way the SWHID is displayed. Setting it to true will include
only the core part of the SWHID in the printed version, and keep the full SWHID, with all contextual
information, in the hyperlink. The default is |false|.

\begin{ltxcode}
  vcs=true|false
\end{ltxcode}

\noindent This option controls the inclusion of the url of the code hosting
platform where the software described in the enttry is developed. The default is |true|.

\subsection{Adding support for additional software identifiers}

It would not be reasonable to have a dedicated field for each of the many software releated identifiers that exist.
If you want to create bibliographic records that use identifiers not natively supported by this package, you
can use the standard Bib\LaTeX\ mechanism that uses the |eprint|, |eprinttype| and |eprintclass| fields.\\

The default formatting of these fields may be what you want, but if it's not the case, you can define
your own format, as explained in the official Bib\LaTeX\ documentation.\\

As an example, this style already contains a specific formatting definition for the Astrophysics Source Code Library (ASCL) software records,
via the following declaration in the \verb|software.bbx| file:

\begin{verbatim}
\DeclareFieldFormat{eprint:ascl}{%
 \mkbibacro{ASCL}\addcolon\addspace%
  \ifhyperref
    {\href{https://ascl.net/#1}{%
         \(\langle\)ascl\addcolon\nolinkurl{#1}\(\rangle\)%
       \iffieldundef{eprintclass}
         {}
         {\addspace\texttt{\mkbibbrackets{\thefield{eprintclass}}}}}}
    {\(\langle\)ascl\addcolon\nolinkurl{#1}\(\rangle\)%
     \iffieldundef{eprintclass}
       {}
       {\addspace\texttt{\mkbibbrackets{\thefield{eprintclass}}}}}
}
\end{verbatim}

If you want to adapt this very example to an identifier |foo| with resolver prefix |https://myfoo.org/|, just replace in the \LaTeX{} code above |https://ascl.net/| with |https://myfoo.org/|, |ascl| with |foo| and |ASCL| with |FOO|.

\section{Details}

The detailed information for this style is contained in the example document and
accompanying \path{.bib} files:
\begin{description}
\item[\path{software-biblatex.tex}] This document.
\item[\path{biblio.bib}] An example bibliography showcasing all the software entries.
\item[\path{sample-use-sty.tex}]\footnote{\path{sample-use-sty.pdf} is also
  provided and is the typeset version of this \latex source file.} This document
  exerces most useful feature of this style extension, using the
  \path{biblio.bib} entries.
\item[\path{sample.tex}] This document produces the same output as
  \path{sample-use-sty.tex}, but instead of extending on the fly and existing
  style, it assumes that an extended bibliographic style \texttt{numeric+sw} has
  been created starting from the standard \texttt{numeric} style.
\item[\path{software.bbx}] The |biblatex-software| references style.
\item[\path{software.dbx}] The |biblatex-software| data model additions.
\item[\path{*.lbx}] The |biblatex-software| localisation files.
\item[\path{software-biblatex.sty}] The |software-biblatex| \LaTeX\ package for extending on the fly any preloaded Bib \LaTeX\ style.
\end{description}

\section{Contributing}

This style extension is currently developed on a git-based repository at
\url{https://gitlab.inria.fr/gt-sw-citation/bibtex-sw-entry/}.
Contributions and bug reports are very welcome. In particular, translation
of the localization strings for other languages are needed.

\section{Revision history}\label{rev}

\begin{changelog}

\begin{release}{bltx-v1.2-6}{2024-12-30} \item Updated URL for swMath links, and minor fixes. \end{release}
\begin{release}{bltx-v1.2-5}{2022-08-02} \item Fix mishandling of SWHID short toggle when hyperref is not loaded \end{release}
\begin{release}{bltx-v1.2-4}{2022-03-03} \item Add support for displaying short SWHID \end{release}
\begin{release}{bltx-v1.2-3}{2021-08-20} \item Support backrefs. \end{release}
\begin{release}{bltx-v1.2-2}{2020-06-27} \item Fix handling of related field; use date instead of year/month in examples; add swMATH definition \end{release}
\begin{release}{bltx-v1.2-1}{2020-06-01} \item Fix mishandling of SWHIDs and HALids when hyperref is not loaded. Fix wrong origins in some SWHIDs in the examples. Improve ASCL example. \end{release}
\begin{release}{bltx-v1.2}{2020-05-29} \item Bump version to 1.2 with clean support of multiline SWHIDs \end{release}
\begin{release}{bltx-v1.1}{2020-04-29} \item Add support for the institution, organization, eprint, eprinttype and eprintclass fields Force urls output when they are the only reference available Updates to the documentation \end{release}
\begin{release}{bltx-v1.0}{2020-04-25} \item First public release \end{release}
\begin{release}{bltx-v0.9}{2020-04-25} \item Preparing for public release: Licence, Readme, update documentation, handle suggestions from the Working Group \end{release}
\begin{release}{bltx-v0.8}{2020-04-09} \item Make the style usable as an extension, and keep possibility of generating extended styles \end{release}
\begin{release}{bltx-v0.7}{2020-04-09} \item Move to diff model approach to be more portable \end{release}
\begin{release}{bltx-v0.6}{2020-04-08} \item Standardise file names, make softwarebib.tex self contained, separate out sample.tex, update Makefile, use printdate macro \end{release}
\begin{release}{bltx-v0.5}{2020-04-08} \item Added standard list format for licenses \end{release}
\begin{release}{bltx-v0.4}{2020-04-07} \item Added repository and licence field \end{release}
\begin{release}{bltx-v0.3}{2020-04-05} \item Biblatex style with first complete example \end{release}
\begin{release}{bltx-v0.2}{2020-04-02} \item Biblatex style sent for review \end{release}
\begin{release}{bltx-v0.1}{2020-04-02} \item First version of the biblatex style \end{release}


\end{changelog}

\bibliography{manual}
\bibliographystyle{abbrv}

\end{document}
